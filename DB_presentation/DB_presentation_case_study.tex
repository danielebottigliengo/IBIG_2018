\documentclass[ignorenonframetext,a4paper]{beamer}
\setbeamertemplate{caption}[numbered]
\setbeamertemplate{caption label separator}{: }
\setbeamercolor{caption name}{fg=normal text.fg}
\beamertemplatenavigationsymbolsempty
\usepackage{lmodern}
\usepackage{amssymb,amsmath}
\usepackage{ifxetex,ifluatex}
\usepackage{fixltx2e} % provides \textsubscript
\ifnum 0\ifxetex 1\fi\ifluatex 1\fi=0 % if pdftex
  \usepackage[T1]{fontenc}
  \usepackage[utf8]{inputenc}
\else % if luatex or xelatex
  \ifxetex
    \usepackage{mathspec}
  \else
    \usepackage{fontspec}
  \fi
  \defaultfontfeatures{Ligatures=TeX,Scale=MatchLowercase}
\fi
\usetheme[]{Padova}
% use upquote if available, for straight quotes in verbatim environments
\IfFileExists{upquote.sty}{\usepackage{upquote}}{}
% use microtype if available
\IfFileExists{microtype.sty}{%
\usepackage{microtype}
\UseMicrotypeSet[protrusion]{basicmath} % disable protrusion for tt fonts
}{}
\newif\ifbibliography
\hypersetup{
            pdfborder={0 0 0},
            breaklinks=true}
\urlstyle{same}  % don't use monospace font for urls
\usepackage{color}
\usepackage{fancyvrb}
\newcommand{\VerbBar}{|}
\newcommand{\VERB}{\Verb[commandchars=\\\{\}]}
\DefineVerbatimEnvironment{Highlighting}{Verbatim}{commandchars=\\\{\}}
% Add ',fontsize=\small' for more characters per line
\usepackage{framed}
\definecolor{shadecolor}{RGB}{248,248,248}
\newenvironment{Shaded}{\begin{snugshade}}{\end{snugshade}}
\newcommand{\KeywordTok}[1]{\textcolor[rgb]{0.13,0.29,0.53}{\textbf{#1}}}
\newcommand{\DataTypeTok}[1]{\textcolor[rgb]{0.13,0.29,0.53}{#1}}
\newcommand{\DecValTok}[1]{\textcolor[rgb]{0.00,0.00,0.81}{#1}}
\newcommand{\BaseNTok}[1]{\textcolor[rgb]{0.00,0.00,0.81}{#1}}
\newcommand{\FloatTok}[1]{\textcolor[rgb]{0.00,0.00,0.81}{#1}}
\newcommand{\ConstantTok}[1]{\textcolor[rgb]{0.00,0.00,0.00}{#1}}
\newcommand{\CharTok}[1]{\textcolor[rgb]{0.31,0.60,0.02}{#1}}
\newcommand{\SpecialCharTok}[1]{\textcolor[rgb]{0.00,0.00,0.00}{#1}}
\newcommand{\StringTok}[1]{\textcolor[rgb]{0.31,0.60,0.02}{#1}}
\newcommand{\VerbatimStringTok}[1]{\textcolor[rgb]{0.31,0.60,0.02}{#1}}
\newcommand{\SpecialStringTok}[1]{\textcolor[rgb]{0.31,0.60,0.02}{#1}}
\newcommand{\ImportTok}[1]{#1}
\newcommand{\CommentTok}[1]{\textcolor[rgb]{0.56,0.35,0.01}{\textit{#1}}}
\newcommand{\DocumentationTok}[1]{\textcolor[rgb]{0.56,0.35,0.01}{\textbf{\textit{#1}}}}
\newcommand{\AnnotationTok}[1]{\textcolor[rgb]{0.56,0.35,0.01}{\textbf{\textit{#1}}}}
\newcommand{\CommentVarTok}[1]{\textcolor[rgb]{0.56,0.35,0.01}{\textbf{\textit{#1}}}}
\newcommand{\OtherTok}[1]{\textcolor[rgb]{0.56,0.35,0.01}{#1}}
\newcommand{\FunctionTok}[1]{\textcolor[rgb]{0.00,0.00,0.00}{#1}}
\newcommand{\VariableTok}[1]{\textcolor[rgb]{0.00,0.00,0.00}{#1}}
\newcommand{\ControlFlowTok}[1]{\textcolor[rgb]{0.13,0.29,0.53}{\textbf{#1}}}
\newcommand{\OperatorTok}[1]{\textcolor[rgb]{0.81,0.36,0.00}{\textbf{#1}}}
\newcommand{\BuiltInTok}[1]{#1}
\newcommand{\ExtensionTok}[1]{#1}
\newcommand{\PreprocessorTok}[1]{\textcolor[rgb]{0.56,0.35,0.01}{\textit{#1}}}
\newcommand{\AttributeTok}[1]{\textcolor[rgb]{0.77,0.63,0.00}{#1}}
\newcommand{\RegionMarkerTok}[1]{#1}
\newcommand{\InformationTok}[1]{\textcolor[rgb]{0.56,0.35,0.01}{\textbf{\textit{#1}}}}
\newcommand{\WarningTok}[1]{\textcolor[rgb]{0.56,0.35,0.01}{\textbf{\textit{#1}}}}
\newcommand{\AlertTok}[1]{\textcolor[rgb]{0.94,0.16,0.16}{#1}}
\newcommand{\ErrorTok}[1]{\textcolor[rgb]{0.64,0.00,0.00}{\textbf{#1}}}
\newcommand{\NormalTok}[1]{#1}
\usepackage{longtable,booktabs}
\usepackage{caption}
% These lines are needed to make table captions work with longtable:
\makeatletter
\def\fnum@table{\tablename~\thetable}
\makeatother
\usepackage{graphicx,grffile}
\makeatletter
\def\maxwidth{\ifdim\Gin@nat@width>\linewidth\linewidth\else\Gin@nat@width\fi}
\def\maxheight{\ifdim\Gin@nat@height>\textheight0.8\textheight\else\Gin@nat@height\fi}
\makeatother
% Scale images if necessary, so that they will not overflow the page
% margins by default, and it is still possible to overwrite the defaults
% using explicit options in \includegraphics[width, height, ...]{}
\setkeys{Gin}{width=\maxwidth,height=\maxheight,keepaspectratio}

% Prevent slide breaks in the middle of a paragraph:
\widowpenalties 1 10000
\raggedbottom

\AtBeginPart{
  \let\insertpartnumber\relax
  \let\partname\relax
  \frame{\partpage}
}
\AtBeginSection{
  \ifbibliography
  \else
    \let\insertsectionnumber\relax
    \let\sectionname\relax
    \frame{\sectionpage}
  \fi
}
\AtBeginSubsection{
  \let\insertsubsectionnumber\relax
  \let\subsectionname\relax
  \frame{\subsectionpage}
}

\setlength{\parindent}{0pt}
\setlength{\parskip}{6pt plus 2pt minus 1pt}
\setlength{\emergencystretch}{3em}  % prevent overfull lines
\providecommand{\tightlist}{%
  \setlength{\itemsep}{0pt}\setlength{\parskip}{0pt}}
\setcounter{secnumdepth}{0}
\usepackage{booktabs}
\usepackage{subfig}
\usepackage{multicol}
\usepackage{rotating}
\usepackage{mathtools}
\usepackage{pgfplots}
\usepackage{listings}
\usepackage{multirow}
\usepackage{amssymb}
\usepackage{pifont}
\usepackage{tikz}
\usepackage{graphics}
\usepackage{hyperref}
\usepackage{enumerate}
\hypersetup{ colorlinks = true, linkcolor = blue, filecolor = magenta, urlcolor = cyan,}

\title{\Large\textbf{Introduction to Bayesian computation and application to regression models and survival analysis}}
\subtitle{\large\textbf{\textrm{IBIG 2018}}}
\author{\centering\underline{\textbf{Daniele Bottigliengo}}\thanks{\tiny Unit of Biostatistics, Epidemiology and Public Health, Department of \newline Cardiac, Thoracic, Vascular Sciences and Public Health, University of Padua, Italy}}
\date{\centering\emph{Padova, Italy, November 22, 2018}}

\begin{document}
\frame{\titlepage}

\section{Survival Analysis Case
Study}\label{survival-analysis-case-study}

\begin{frame}{Survival Ovarian Cancer}

\begin{itemize}
\setlength\itemsep{1em}
  \item{Randomized trial comparing treatment of patients with advanced
        ovarian carcinoma (stages $IIIB$ and $IV$)}
  \item{Two groups of patients:}
  \begin{itemize}
    \item{Cyclophosphamide alone ($1 \> g/m^{2}$)}
    \item{Cyclophosphamide ($500 \> \mu g/m^{2}$) plus Adriamycin 
        ($40 \> \mu g/m2$)}
  \end{itemize}
\item{Intravenous (IV) injection every $3$ weeks}
\end{itemize}

\end{frame}

\begin{frame}{The dataset (1)}

\begin{itemize}
\setlength\itemsep{1em}
  \item{$26$ women enrolled}
  \item{The following information were retrieved:}
  \begin{itemize}
    \item{Age}
    \item{Presence of residual disease}
    \item{ECOG performance}
    \item{Median follow-up time in the Cyclophosphamide group: 
          $448$ days}
    \item{Median follow-up time in the Cyclophosphamide plus 
          Adriamycin $563$ days}
  \end{itemize}
\item{$12$ patients died during the study and $14$ were right-censored}
\end{itemize}

\end{frame}

\begin{frame}{The dataset (2)}

\tiny

\begin{longtable}[]{@{}rlrllr@{}}
\toprule
follow\_up\_days & status & age & residual\_disease & treatment &
ecog\_performance\tabularnewline
\midrule
\endhead
59 & dead & 72.3315 & yes & Cyclo & 1\tabularnewline
115 & dead & 74.4932 & yes & Cyclo & 1\tabularnewline
156 & dead & 66.4658 & yes & Cyclo & 2\tabularnewline
421 & alive & 53.3644 & yes & Cyclo + Adria & 1\tabularnewline
431 & dead & 50.3397 & yes & Cyclo & 1\tabularnewline
448 & alive & 56.4301 & no & Cyclo & 2\tabularnewline
464 & dead & 56.9370 & yes & Cyclo + Adria & 2\tabularnewline
475 & dead & 59.8548 & yes & Cyclo + Adria & 2\tabularnewline
477 & alive & 64.1753 & yes & Cyclo & 1\tabularnewline
563 & dead & 55.1781 & no & Cyclo + Adria & 2\tabularnewline
\bottomrule
\end{longtable}

\end{frame}

\begin{frame}{Exploratory data analysis (1)}

\includegraphics{DB_presentation_case_study_files/figure-beamer/unnamed-chunk-3-1.pdf}

\end{frame}

\begin{frame}{Exploratory data analysis (2)}

\includegraphics{DB_presentation_case_study_files/figure-beamer/unnamed-chunk-4-1.pdf}

\end{frame}

\begin{frame}{Exploratory data analysis (3)}

\includegraphics{DB_presentation_case_study_files/figure-beamer/unnamed-chunk-5-1.pdf}

\end{frame}

\begin{frame}{Survival Model}

Weibull parametric proportional hazard model: \[
  f \left( t \vert \alpha, \sigma \right) = \frac{\alpha}{\sigma} \left( \frac{t}{\sigma}\right)^{\alpha - 1} e ^{ - \left( \frac{t}{\sigma} \right) }
  \] where:

\begin{itemize}
 \item{$\alpha$ is the shape parameter}
 \item{$\sigma$ is the scale parameter, where $\sigma = e ^{ - \left( \frac{\eta}{\alpha} \right) }$}.
 \item{$\eta$ is the linear predictor and it can be expressed as 
       function of some covariates}
\end{itemize}

\end{frame}

\begin{frame}{Fake data simulations}

\begin{itemize}
  \item{Starting point of model fitting}
  \item{Check if the model makes sense}
\end{itemize}

\begin{enumerate}
  \item{Simulate fake data from the prior predictive distributions}
  \item{Fit the model to the simulated data}
  \item{Are true parameters values included in the posterior 
        distributions?}
\end{enumerate}

\end{frame}

\begin{frame}[fragile]{The model: data block}

\scriptsize

\begin{Shaded}
\begin{Highlighting}[]
\StringTok{"}
\StringTok{data \{}

\StringTok{  int<lower = 0> n_obs;             // Number of deaths}
\StringTok{  int<lower = 0> n_cens;            // Number of censored}
\StringTok{  vector[n_obs] y_obs;              // Death vector}
\StringTok{  vector[n_cens] y_cens;            // Censored vector}
\StringTok{  int<lower = 0> k;                 // Number of covariates}
\StringTok{  matrix[n_obs, k] x_obs;           // Design matrix for deaths}
\StringTok{  matrix[n_cens, k] x_cens;         // Design matrix for censoring}

\StringTok{\}}

\StringTok{transformed data \{}

\StringTok{  real<lower = 0> tau_beta_0;       // Sd of intercept}
\StringTok{  real<lower = 0> tau_alpha;        // Sd alpha}

\StringTok{  tau_beta_0 = 10;}
\StringTok{  tau_alpha = 10;}

\StringTok{\}}
\StringTok{"}
\end{Highlighting}
\end{Shaded}

\end{frame}

\begin{frame}[fragile]{The model: parameters block}

\scriptsize

\begin{Shaded}
\begin{Highlighting}[]
\StringTok{"}
\StringTok{parameters \{}

\StringTok{  real<lower = 0> alpha;           // Alpha parameter on the log scale}
\StringTok{  real beta_0;                     // Intercept}
\StringTok{  vector[k] beta;                  // Coefficients of covariates}

\StringTok{\}}
\StringTok{"}
\end{Highlighting}
\end{Shaded}

\end{frame}

\begin{frame}[fragile]{The model: model block}

\scriptsize

\begin{Shaded}
\begin{Highlighting}[]
\StringTok{"}
\StringTok{model \{}

\StringTok{  // Linear predictors}
\StringTok{  vector[n_obs] eta_obs = beta_0 + x_obs * beta;}
\StringTok{  vector[n_cens] eta_cens = beta_0 + x_cens * beta;}

\StringTok{  // Define the priors}
\StringTok{  target += normal_lpdf(alpha | 0, tau_alpha) +}
\StringTok{            normal_lpdf(beta_0 | 0, tau_beta_0) +}
\StringTok{            normal_lpdf(beta | 0, 1);}

\StringTok{  // Define the likelihood}
\StringTok{  target += weibull_lpdf(y_obs | alpha, exp(-eta_obs/alpha)) +}
\StringTok{            weibull_lccdf(y_cens | alpha, exp(-eta_cens/alpha));}

\StringTok{\}}
\StringTok{"}
\end{Highlighting}
\end{Shaded}

\end{frame}

\begin{frame}{Recover the parameters values}

\includegraphics{DB_presentation_case_study_files/figure-beamer/unnamed-chunk-9-1.pdf}

\end{frame}

\begin{frame}{Fit the model to the real data}

\begin{itemize}
\setlength\itemsep{1em}
  \item{If the fitted model is able to recover the true parameters
        values it is possible to proceed by fitting the model to
        real data}
  \item{Prior Predictive checks can be very useful to question about
        the correctness of the model}
  \item{Before fitting the model to the real data, centering and scale
        the covariates is useful to ease the sampling process}
\end{itemize}

Two steps are important to evaluate the robustness of the analysis:

\begin{itemize}
  \item{MCMC diagnostics}
  \item{Posterior Predictive Checks}
\end{itemize}

\end{frame}

\begin{frame}{MCMC diagnostics: \(R_{hat}\) and \(ESS\)}

\includegraphics{DB_presentation_case_study_files/figure-beamer/unnamed-chunk-10-1.pdf}

\end{frame}

\begin{frame}{MCMC diagnostics: traceplot}

\includegraphics{DB_presentation_case_study_files/figure-beamer/unnamed-chunk-11-1.pdf}

\end{frame}

\begin{frame}{Posterior Predictive Checks (1)}

\includegraphics{DB_presentation_case_study_files/figure-beamer/unnamed-chunk-12-1.pdf}

\end{frame}

\begin{frame}{Posterior Predictive Checks (2)}

\includegraphics{DB_presentation_case_study_files/figure-beamer/unnamed-chunk-13-1.pdf}

\end{frame}

\begin{frame}{Posterior Predictive Checks (3)}

\includegraphics{DB_presentation_case_study_files/figure-beamer/unnamed-chunk-14-1.pdf}

\end{frame}

\begin{frame}{Revise the model}

\begin{itemize}
  \item{The model predicts greater follow-up times than those observed
        in the ovarian cancer data}
  \item{Weibull distribution may not be the best one to model 
        time-to-deaths of subjects with ovarian cancer}
  \item{Different family distributions can be considered, e.g.
        log-normal, gamma, ...}
\end{itemize}

\end{frame}

\begin{frame}{Log-normal (1)}

\includegraphics{DB_presentation_case_study_files/figure-beamer/unnamed-chunk-15-1.pdf}

\end{frame}

\begin{frame}{Log-normal (2)}

\includegraphics{DB_presentation_case_study_files/figure-beamer/unnamed-chunk-16-1.pdf}

\end{frame}

\begin{frame}{Log-normal (3)}

\includegraphics{DB_presentation_case_study_files/figure-beamer/unnamed-chunk-17-1.pdf}

\end{frame}

\begin{frame}{Gamma (1)}

\includegraphics{DB_presentation_case_study_files/figure-beamer/unnamed-chunk-18-1.pdf}

\end{frame}

\begin{frame}{Gamma (2)}

\includegraphics{DB_presentation_case_study_files/figure-beamer/unnamed-chunk-19-1.pdf}

\end{frame}

\begin{frame}{Gamma (3)}

\includegraphics{DB_presentation_case_study_files/figure-beamer/unnamed-chunk-20-1.pdf}

\end{frame}

\begin{frame}{Compare the models (1)}

\begin{itemize}
\setlength\itemsep{1em}
  \item{None of the models seems to greatly improve the fitting of the
        data}
  \item{Models can be compared by using leave-one-out cross-validation
        (LOO-CV)}
  \item{Expected log predictive density (ELPD) computed with LOO-CV
        can be used to evaluate which model has a better fit}
  \item{Predictive weights can be assigned to each model by using
        Stacking, Pseudo bayesian-model-averaging (Pseudo-BMA)}
  \item{Higher ELPD and predictive weights suggest better 
        predictive performances}
\end{itemize}

\end{frame}

\begin{frame}{Compare the models (2)}

\scriptsize

\begin{longtable}[]{@{}lrrr@{}}
\caption{Comparison of ELPD of the fitted models.}\tabularnewline
\toprule
model & elpd\_diff & elpd\_loo & se\_elpd\_loo\tabularnewline
\midrule
\endfirsthead
\toprule
model & elpd\_diff & elpd\_loo & se\_elpd\_loo\tabularnewline
\midrule
\endhead
lognormal & 0.00 & -23.95 & 3.13\tabularnewline
gamma & -1.28 & -25.23 & 3.27\tabularnewline
weibull & -4.02 & -27.97 & 3.40\tabularnewline
\bottomrule
\end{longtable}

\begin{longtable}[]{@{}lrrr@{}}
\caption{Model comparison with Stacking, Pseudo-BMA and Pseudo-BMA with
Bayesian Bootstrap.}\tabularnewline
\toprule
model & stacking & pseudo\_bma & pseudo\_bma\_bb\tabularnewline
\midrule
\endfirsthead
\toprule
model & stacking & pseudo\_bma & pseudo\_bma\_bb\tabularnewline
\midrule
\endhead
weibull & 0 & 0.014 & 0.049\tabularnewline
lognormal & 1 & 0.772 & 0.734\tabularnewline
gamma & 0 & 0.214 & 0.217\tabularnewline
\bottomrule
\end{longtable}

\end{frame}

\begin{frame}{Parameters of the model (1)}

\includegraphics{DB_presentation_case_study_files/figure-beamer/unnamed-chunk-22-1.pdf}

\end{frame}

\begin{frame}{Parameters of the model (2)}

\includegraphics{DB_presentation_case_study_files/figure-beamer/unnamed-chunk-23-1.pdf}

\end{frame}

\begin{frame}{Posterior predictive survival curves}

\includegraphics{DB_presentation_case_study_files/figure-beamer/unnamed-chunk-24-1.pdf}

\end{frame}

\end{document}
